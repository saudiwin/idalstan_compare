%\input{|"curl -L 'https://virginia.box.com/shared/static/9120uqo655t0kwesti7t3exa7zsgbwdq.tex'"}
% Bob's template header for general articles and outlines

\documentclass[12pt]{article}
\fontfamily{times}
\usepackage{url}
\usepackage[margin=1in]{geometry}
\usepackage{array}
\usepackage{ragged2e}

\usepackage{fullpage}
\usepackage{parskip}
\usepackage{titling}
\usepackage{tikz}
\setlength{\parindent}{15pt}
\usepackage{times}
\usepackage{booktabs}
\usepackage{siunitx}
\usepackage{graphicx}
\usepackage{amsmath}
\usepackage{authblk}
\usepackage[american]{babel}
\usepackage{csquotes}
\usepackage[hidelinks]{hyperref}
\usepackage[authordate,backend=biber]{biblatex-chicago}

\addbibresource[location=remote]{https://virginia.box.com/shared/static/m0nw01c4bcltkauwy2hd1467vu7dufrj.bib}
\usepackage[section]{placeins}

\author{Robert Kubinec}
\affil{\small Department of Politics \\
\small University of Virginia}
\date{\small \today}

\title{Appendix for Absence Makes the Ideal Points Sharper: Full-data IRT Models for Legislatures}
\usepackage{amsmath,amsthm, amssymb, latexsym}
\linespread{1.5}

\begin{document}
	\maketitle
	\section*{Complete Exposition of Absence-Inflated IRT}
	
 To write out the model fully, we need to address the four possible outcomes $y_{ijkr}$ for each legislator $i$ and bill $j$: $\{N,A_b,Y,A_s\}$ where $Y$ stands for yes votes, $A_b$ for abstentions, $N$ for no votes, and $A_s$ for absences. I denote the probability that a legislator is absent $P_r$. We can the probability $f(y_{ijkr})$ of each the four outcomes as functions of $P_r$ and the probabilities of two of the three possible votes, $P_N$, $P_{Ab}$, where the third vote $P_Y$ is the residual probability $1 - P_N - P_{Ab}$.
 
 \[
 f(y_{ijkr}) = \begin{cases}
 P_r \text{ if } y_{ijkr}=A_s,\\
 (1-P_r)P_N \text{ if } y_{ijkr}=N,\\
 (1-P_r)P_{Ab} \text{ if } y_{ijkr}=A_b\\
 (1-P_r)(1-P_N-P_{Ab}) \text{ if } y_{ijkr} = Y\\
 \end{cases}
 \]
 
 This version of the model shows clearly how the probability of absence $P_r$ deflates the probabilities of each of the vote outcomes equally. As the probability of absence increases, the probability of each vote outcome falls proportionally. We can combine these expressions into a single likelihood by denoting the outcomes by an indicator function $I$ subscripted with the category and multiplying over the legislators and votes:
 
 \[
 L(P_r,P_N,P_{Ab},P_Y|y_{ijkr}) = \prod_{i=1}^{I} \prod_{j=1}^{J} P_r^{I_{As}} \big((1-P_r)P_N \big)^{I_N} \big((1-P_r)P_{Ab} \big)^{I_{Ab}} \big((1-P_r)(1-P_N-P_{Ab}\big)^{I_Y}
 \]
 
 We then substitute the probabilities $P_r$, $P_N$, $P_{Ab}$, and $P_Y$ with logistic link functions to incorporate the IRT model for each specific category to show the summation over $i$ and $j$. To do so we also incorporate cutpoints $c_1$ and $c_2$ that represent the ordered structure of the votes $N$, $A_b$ and $Y$. 
 
 \begin{align*}
 	 P_r &= \frac{1}{1 + e^{x_i'\gamma_j - \omega_j}}\\
 	P_N &= 1 - \frac{1}{1 + e^{x_i'\alpha_j - \beta_j}}\\
 	P_{Ab} &= \Big( \frac{1}{1 + e^{x_i'\alpha_j - \beta_j}} - c_2 \Big) - \Big( \frac{1}{1 + e^{x_i'\alpha_j - \beta_j}} - c_1 \Big) \\
 	P_Y &= \frac{1}{1 + e^{x_i'\alpha_j - \beta_j}} - c_2 \\
 \end{align*}

This formulation makes it very clear how the cutpoints affect the model. The abstention category is specifically the difference between the two cutpoints, while the yes category is all the probability mass above the higher cutpoint $c_2$. The no category is the remainder after the probabilities of abstention and yes votes have been calculated.

 
\end{document}

