%\input{|"curl -L 'https://virginia.box.com/shared/static/9120uqo655t0kwesti7t3exa7zsgbwdq.tex'"}
% Bob's template header for general articles and outlines

\documentclass[12pt]{article}
\fontfamily{times}
\usepackage{url}
\usepackage[margin=1in]{geometry}
\usepackage{array}
\usepackage{ragged2e}

\usepackage{fullpage}
\usepackage{parskip}
\usepackage{titling}
\usepackage{tikz}
\setlength{\parindent}{15pt}
\usepackage{times}
\usepackage{booktabs}
\usepackage{siunitx}
\usepackage{graphicx}
\usepackage{amsmath}
\usepackage{authblk}
\usepackage[american]{babel}
\usepackage{csquotes}
\usepackage[hidelinks]{hyperref}
\usepackage[authordate,backend=biber]{biblatex-chicago}

\addbibresource[location=remote]{https://virginia.box.com/shared/static/m0nw01c4bcltkauwy2hd1467vu7dufrj.bib}
\usepackage[section]{placeins}

\author{Robert Kubinec}
\affil{\small Department of Politics \\
\small University of Virginia}
\date{\small \today}

\title{Absence Makes the Ideal Points Sharper: Full-data IRT Models for Legislatures}
\usepackage{amsmath,amsthm, amssymb, latexsym}
\linespread{1.5}
\begin{document}
	
	\maketitle
	
	\begin{abstract}
		I put forward a new approach for incorporating missing data in roll call vote estimation by extending existing approaches to account for legislators' strategic decision to turn up to vote. This estimation uses the concept of a hurdle model to deflate the probabilities of legislators' votes by the probability of absences. The model produces a single set of ideal points, but utilizes different parameters to allow the bills to have a latent score for the likelihood of absence given legislators' ideology. Compared to existing approaches, this model tends to produce more accurate estimates of ideal points for US Congresspeople who have higher rates of absence, such as senators on the campaign trail. For parliamentary data, the model provides much more precise estimates, especially in legislatures with very high rates of absence as can occur in emerging democracies. Additionally, the model can incorporate absentions as a middle category between yes and no votes for legislatures with significant rates of abstentions. This model is easily accessible through the R package \texttt{idealstan}.
	\end{abstract}
	
	
\end{document}